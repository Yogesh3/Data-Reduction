\documentclass[12pt]{article}
%Packages
\usepackage[top= 1.0 in, bottom= 1.0in, left= 1.0in, right= 1.0 in]{geometry}
\usepackage{amsmath}
\usepackage{graphicx}
\usepackage{indentfirst}
\usepackage{url}
%Non-Standard Commands
\renewcommand\thesection{\Roman{section}}  %number sections with roman numerals
\providecommand{\e}[1]{\ensuremath{\times 10^{#1}}} %scientific notation
\newcommand{\iu}{{i\mkern1mu}}
\begin{document}
%Title
\title{Lab 1: Exoplanet Transit}
\author{Authors: Joseph Monroy and Yogesh Mehta \\ AST 443: Observational Techniques in Astronomy}
\date{Performed on October 14-15, 2016 \linebreak \linebreak Submitted on \today}
\maketitle
\section*{Abstract}
Our goal for this experiment was to successfully detect a Jupiter-type exoplanet, and from this to see whether it has the known mid-transit time and transit duration. Also, from the depth of the transit we want to get the ratio between the radii of the planet and the star. In our experiment we choose to view the star HD 209458 while its planet HD 209458b, also known as Osiris, was transiting. 

\section{Introduction}
It has only been within the past decade and a half that we have been able to detect planets around other stars. This has been a major breakthrough in all fields of astronomy because it allows us to see what kind of planets form around different types of stars. Also, it allows us to speculate more on whether we are alone in the universe or not. The most recently discovered method used to detect exoplanets is through transit photometry. This method allows us to point our telescope to a star and see if theres a change in observed flux. Depending on the duration of the observed flux from the star we can determine if there is a transiting body. 

This is the method we used in our experiment and it has proven very successful over the years for hundreds of other astronomers. Although it has already been confirmed that there is an exoplanet around HD 209458, our experiment was significant because it allowed us to see first hand how powerful the method of transit photometry is. By just knowing the dip in relative flux, we can determine the ratio between the radii of the planet and star. Then from knowing the radius of the host star, we can determine the radius of the exoplanet. This is the reason why this method is so powerful: it allows us to learn the properties of planets hundreds of light year without even viewing or seeing that they're there. 

\newpage
\section{Observations} 
wkjlkjfsdkljf





\end{document}