
\documentclass{aastex61}
\usepackage{graphicx}
%% The default is a single spaced, 10 point font, single spaced article.
%% There are 5 other style options available via an optional argument. They
%% can be envoked like this:
%%
%% \documentclass[argument]{aastex61}
%% 
%% where the arguement options are:
%%
%%  twocolumn   : two text columns, 10 point font, single spaced article.
%%                This is the most compact and represent the final published
%%                derived PDF copy of the accepted manuscript from the publisher
%%  manuscript  : one text column, 12 point font, double spaced article.
%%  preprint    : one text column, 12 point font, single spaced article.  
%%  preprint2   : two text columns, 12 point font, single spaced article.
%%  modern      : a stylish, single text column, 12 point font, article with
%% 		  wider left and right margins. This uses the Daniel
%% 		  Foreman-Mackey and David Hogg design.
%%
%% Note that you can submit to the AAS Journals in any of these 6 styles.
%%
%% There are other optional arguments one can envoke to allow other stylistic
%% actions. The available options are:
%%
%%  astrosymb    : Loads Astrosymb font and define \astrocommands. 
%%  tighten      : Makes baselineskip slightly smaller, only works with 
%%                 the twocolumn substyle.
%%  times        : uses times font instead of the default
%%  linenumbers  : turn on lineno package.
%%  trackchanges : required to see the revision mark up and print its output
%%  longauthor   : Do not use the more compressed footnote style (default) for 
%%                 the author/collaboration/affiliations. Instead print all
%%                 affiliation information after each name. Creates a much
%%                 long author list but may be desirable for short author papers
%%
%% these can be used in any combination, e.g.
%%
%% \documentclass[twocolumn,linenumbers,trackchanges]{aastex61}

%% AASTeX v6.* now includes \hyperref support. While we have built in specific
%% defaults into the classfile you can manually override them with the
%% \hypersetup command. For example,
%%
%%\hypersetup{linkcolor=red,citecolor=green,filecolor=cyan,urlcolor=magenta}
%%
%% will change the color of the internal links to red, the links to the
%% bibliography to green, the file links to cyan, and the external links to
%% magenta. Additional information on \hyperref options can be found here:
%% https://www.tug.org/applications/hyperref/manual.html#x1-40003






%% This is the end of the preamble.  Indicate the beginning of the
%% manuscript itself with \begin{document}.

\begin{document}

\title{Lab 1: Exoplanet Transit \\ AST 443: Observational Techniques in Astronomy}

\date{Performed October 7-8, 2016 and Submitted \today}

\author{Joseph Monroy}

\author{Yogesh Mehta}


\begin{abstract}
Our goal for this experiment was to successfully detect a Jupiter-type exoplanet, and from this to see whether it has the known mid-transit time and transit duration. Also, from the depth of the transit we want to get the ratio between the radii of the planet and the star. In our experiment we choose to view the star HD 209458 while its planet HD 209458b, also known as Osiris, was transiting. 
\end{abstract}

\section{Introduction (JM)} \label{sec:intro}
It has only been within the past decade and a half that we have been able to detect planets around other stars. This has been a major breakthrough in all fields of astronomy because it allows us to see what kind of planets form around different types of stars. Also, it allows us to speculate more on whether we are alone in the universe or not. The most recently discovered method used to detect exoplanets is through transit photometry. This method allows us to point our telescope to a star and see if theres a change in observed flux. Depending on the duration of the observed flux from the star we can determine if there is a transiting body. 

This is the method we used in our experiment and it has proven very successful over the years for hundreds of other astronomers. Although it has already been confirmed that there is an exoplanet around HD 209458, our experiment was significant because it allowed us to see first hand how powerful the method of transit photometry is. By just knowing the dip in relative flux, we can determine the ratio between the radii of the planet and star. Then from knowing the radius of the host star, we can determine the radius of the exoplanet. This is the reason why this method is so powerful: it allows us to learn the properties of planets hundreds of light years away without even viewing or seeing the planet. 

\section{Observations (JM)}
We decided to observe HD 209458 because of the unique characteristics of the system. The star is of G type making it very similar to our own star, having a mass of 1.148 $M_\sun$ and a radius of 1.203 $R_\sun$. The planet has a radius 1.38 times that of Jupiter, while orbiting and nearly one eighth the orbital radius of Mercury to the Sun. This makes Osiris a hot Jupiter type planet because its the size of Jupiter but extremely close to it's star. This type of system is a great candidate for exoplanet detection through transit photometry. This is because the star is of average size, while having a huge Jupiter size planet orbiting it. This makes detecting a dip in the relative flux when the planet transits a lot easier. The only negative about this system is the orbital radius of Osiris, because the closer a planet is to its star the harder it will be to detect it. Although it makes up for it by the size of Osiris and the relatively bright star of 7.65 apparent magnitude.

We were able to observe HD 209458 and its transiting planet by using a 14-inch Meade LX200-ACF telescope with a SBIG STL-1001E CCD camera attached. This yields a field of view of 26 arcminutes for the exposures. All observations were held at the Mt. Stony Brook Observatory. The weather that night was mostly clear with some thin overcast clouds in the beginning of observations. By the end of the experiment the sky was clear of clouds. The seeing was mostly good throughout the night as well with there being very little wind. We calculated the total transit duration to be 3.54 hours, with a mid-transit time of October $8^{\text{th}}$, 2016 at 02:23:23 UTC. The transit duration was calculated using the following equation:
\begin{equation}
T_{duration} = \frac{P}{\pi}\arcsin\left(\frac{\sqrt{(R_*+R_P)^{2}-(bR_*)^{2}}}{a}\right)
\end{equation}
Where $P$, $a$ and $b$ are all the period, semi-major axis and impact parameter of the planet respectively, and $R_*$ and $R_P$ are the radii of the star and planet respectively. The mid-transit time was calculated from knowing the period of the planet, and knowing a specific time when the transit began in the past. From this information we simply just have to keep adding the period time to the time of transit until we reach current day. We figured out where our star should be in the night sky by using the program StarAlt. It helps show whether stars are viewable depending on your location on Earth and your elevation.

We first tried to find the star by going to it's right accession and declination, and made sure that the star we found was HD 209458 by looking at our finder charts. Once we confirmed the star we made sure it was centered on the cross hairs of the eyepiece. We then made sure the CCD camera was connected correctly to the telescope and the program CCDSoft, which is used for adjusting the exposure times and other settings of the CCD. We first tried to take four exposures of our star with an exposure time (exp. time) of three seconds each. This resulted in counts that were too high because the star was focused too well on the CCDSoft. High counts are bad because the pixels on the CCD can be oversaturated at around 60,000 counts and not yield good data. So for the next exposure we defocused the star, while increasing the exposure time to five seconds, which resulted in lower counts around 20,000. We kept this exposure time for the rest of the science exposures. By the end of the experiment we took 1,461 science exposures, all with an exposure time of five seconds. After this we started to take our calibration data, starting with the darks. In order to take the darks we put the telescope cap back on so we have complete darkness. We then took 19 dark exposures each of five seconds, then we took out flats. We took the cap off and pointed the telescope towards a equally lit part of the dome. At first we took 19 flats each of five seconds, but it turned out the counts were too high. So we lowered the exposure time to three seconds and lowered the lights in the observatory. This yielded in lower counts and better flat fields. Finally we took another set of flats with a shorter exposure time of one second. This is so later we can create a bad pixel mask so we know where the bad pixels are on the CCD. 


\section{Data Reduction (JM)}
The data we obtained from all the exposures were in the form of raw .FIT files. So in order to make the science .FIT files usable we needed to create a Master Flat field, a Master Dark field and a bad pixel mask from all the calibration data. This was done using the Python code called (insert here), which created clean, calibrated science images. All of data reduction was done through the uhura astronomy computer using a secure shell client. When we performed the experiment the auto-guider on the telescope did not work, so we had to use the astronmetry.net software to get the position and coordinates of all the stars in our .FIT files. We wrote script called (insert here) to let the software run through all of our images. Next, we used the program Source Extractor to return the properties of all the stars in our images, most importantly the relative flux. However, we needed to know the aperture size Source extractor should use to get the data from the stars. This could be found out by opening up one of our clean images in ds9 and seeing what the radius in pixels is our target star. Once we found this radius, we inputed it into the Source Extractor configuration file and obtained the properties of every object in all of our images. We once again wrote a script called (insert here) that runs Source Extractor through all our images.

We finally now have the three properties we need to get our analyzed data: the time of each image, flux received from all the stars in one image, and the error in that flux. Next we choose ten reference stars in of our clean images and took their respective times, flux, and the error in the flux from the table. We did this because it allows us to calibrate the flux received from the target star. All of these stars have similar brightness when we viewed them in ds9. From the fluxes of each reference star we rescaled them to by the star's average flux over all exposures taken. This was done also to the error in the flux, shown here:
\begin{equation}
f_{j}(t)  = \frac{f_{j}(t)}{\langle f_{j} \rangle}       \;\;\;\;\;\;\;\sigma_{f_{j}(t)} = \frac{\sigma_{f_{j}(t)}}{\langle f_{j} \rangle}
\end{equation}
Where $f_{j}(t)$ is the flux for some reference $j$ at time $t$ and $\sigma_{f_{j}(t)}$ is the error in that particular flux. We normalized the flux and its error because it would allow us to visualize any changes in airmass or atmospheric turbulence during the observation. Once we had these values calculated from the equation (2) we plotted each the flux as a function of time. The light curves for all ten reference stars are shown below:

%Include 10  ref star light curves
%\begin{figure}[htbp]
%    \includegraphics[scale = 0.5]{}
%    \label{fig:refcurve1}
%    \caption{}
%\end{figure}
From these curves we can see vast changes in flux caused by the atmospheric conditions that night. Now that we know that they are present we can account for in this in our the flux from the target star by calculating it relative to the flux of the reference stars. In order to get all the flux from all the reference stars we need to calculate the weighted mean in each clean image. This can be found by using:
\begin{equation}
	\mu_{i}^{ref} = \frac{\sum_{j}\left( \frac{f_{j}^{ref}}{\left( \sigma_{j}^{ref}\right) ^{2}}\right) }{\sum_{j}\left( \frac{1}{\left( \sigma_{j}^{ref}\right) ^{2}}\right) }
\end{equation}
Where $\mu_{i}^{ref}$ is the weighted mean for any image $i$. This weighted mean also has an error given by:
\begin{equation}
	\sigma_{i}^{ref} = \sqrt{\frac{1}{\sum_{j}\left( \frac{1}{\left( \sigma_{j}^{ref}\right) ^{2}}\right)}}
\end{equation}

\section{Data Analysis and Results}
\section{Discussion}
\section{Conclusion}

\end{document}