%%
%% Beginning of file 'sample61.tex'
%%
%% Modified 2016 September
%%
%% This is a sample manuscript marked up using the
%% AASTeX v6.1 LaTeX 2e macros.
%%
%% AASTeX is now based on Alexey Vikhlinin's emulateapj.cls 
%% (Copyright 2000-2015).  See the classfile for details.

%% AASTeX requires revtex4-1.cls (http://publish.aps.org/revtex4/) and
%% other external packages (latexsym, graphicx, amssymb, longtable, and epsf).
%% All of these external packages should already be present in the modern TeX 
%% distributions.  If not they can also be obtained at www.ctan.org.

%% The first piece of markup in an AASTeX v6.x document is the \documentclass
%% command. LaTeX will ignore any data that comes before this command. The 
%% documentclass can take an optional argument to modify the output style.
%% The command below calls the preprint style  which will produce a tightly 
%% typeset, one-column, single-spaced document.  It is the default and thus
%% does not need to be explicitly stated.
%%
%%
%% using aastex version 6.1
\documentclass{aastex61}

%% The default is a single spaced, 10 point font, single spaced article.
%% There are 5 other style options available via an optional argument. They
%% can be envoked like this:
%%
%% \documentclass[argument]{aastex61}
%% 
%% where the arguement options are:
%%
%%  twocolumn   : two text columns, 10 point font, single spaced article.
%%                This is the most compact and represent the final published
%%                derived PDF copy of the accepted manuscript from the publisher
%%  manuscript  : one text column, 12 point font, double spaced article.
%%  preprint    : one text column, 12 point font, single spaced article.  
%%  preprint2   : two text columns, 12 point font, single spaced article.
%%  modern      : a stylish, single text column, 12 point font, article with
%% 		  wider left and right margins. This uses the Daniel
%% 		  Foreman-Mackey and David Hogg design.
%%
%% Note that you can submit to the AAS Journals in any of these 6 styles.
%%
%% There are other optional arguments one can envoke to allow other stylistic
%% actions. The available options are:
%%
%%  astrosymb    : Loads Astrosymb font and define \astrocommands. 
%%  tighten      : Makes baselineskip slightly smaller, only works with 
%%                 the twocolumn substyle.
%%  times        : uses times font instead of the default
%%  linenumbers  : turn on lineno package.
%%  trackchanges : required to see the revision mark up and print its output
%%  longauthor   : Do not use the more compressed footnote style (default) for 
%%                 the author/collaboration/affiliations. Instead print all
%%                 affiliation information after each name. Creates a much
%%                 long author list but may be desirable for short author papers
%%
%% these can be used in any combination, e.g.
%%
%% \documentclass[twocolumn,linenumbers,trackchanges]{aastex61}

%% AASTeX v6.* now includes \hyperref support. While we have built in specific
%% defaults into the classfile you can manually override them with the
%% \hypersetup command. For example,
%%
%%\hypersetup{linkcolor=red,citecolor=green,filecolor=cyan,urlcolor=magenta}
%%
%% will change the color of the internal links to red, the links to the
%% bibliography to green, the file links to cyan, and the external links to
%% magenta. Additional information on \hyperref options can be found here:
%% https://www.tug.org/applications/hyperref/manual.html#x1-40003

%% If you want to create your own macros, you can do so
%% using \newcommand. Your macros should appear before
%% the \begin{document} command.
%%
\newcommand{\vdag}{(v)^\dagger}
\newcommand\aastex{AAS\TeX}
\newcommand\latex{La\TeX}



%% Mark up commands to limit the number of authors on the front page.
%% Note that in AASTeX v6.1 a \collaboration call (see below) counts as
%% an author in this case.
%
%\AuthorCollaborationLimit=3
%
%% Will only show Schwarz, Muench and "the AAS Journals Data Scientist 
%% collaboration" on the front page of this example manuscript.
%%
%% Note that all of the author will be shown in the published article.
%% This feature is meant to be used prior to acceptance to make the
%% front end of a long author article more manageable. Please do not use
%% this functionality for manuscripts with less than 20 authors. Conversely,
%% please do use this when the number of authors exceeds 40.
%%
%% Use \allauthors at the manuscript end to show the full author list.
%% This command should only be used with \AuthorCollaborationLimit is used.

%% The following command can be used to set the latex table counters.  It
%% is needed in this document because it uses a mix of latex tabular and
%% AASTeX deluxetables.  In general it should not be needed.
%\setcounter{table}{1}

%%%%%%%%%%%%%%%%%%%%%%%%%%%%%%%%%%%%%%%%%%%%%%%%%%%%%%%%%%%%%%%%%%%%%%%%%%%%%%%%
%%%%%%%%%%%%%%%%%%%%%%%%%

%% This is the end of the preamble.  Indicate the beginning of the
%% manuscript itself with \begin{document}.

\begin{document}

\title{Lab 1: Exoplanet Transit \\ AST 443: Observational Techniques in Astronomy}

%% LaTeX will automatically break titles if they run longer than
%% one line. However, you may use \\ to force a line break if
%% you desire. In v6.1 you can include a footnote in the title.

%% A significant change from earlier AASTEX versions is in the structure for 
%% calling author and affilations. The change was necessary to implement 
%% autoindexing of affilations which prior was a manual process that could 
%% easily be tedious in large author manuscripts.
%%
%% The \author command is the same as before except it now takes an optional
%% arguement which is the 16 digit ORCID. The syntax is:
%% \author[xxxx-xxxx-xxxx-xxxx]{Author Name}
%%
%% This will hyperlink the author name to the author's ORCID page. Note that
%% during compilation, LaTeX will do some limited checking of the format of
%% the ID to make sure it is valid.
%%
%% Use \affiliation for affiliation information. The old \affil is now aliased
%% to \affiliation. AASTeX v6.1 will automatically index these in the header.
%% When a duplicate is found its index will be the same as its previous entry.
%%
%% Note that \altaffilmark and \altaffiltext have been removed and thus 
%% can not be used to document secondary affiliations. If they are used latex
%% will issue a specific error message and quit. Please use multiple 
%% \affiliation calls for to document more than one affiliation.
%%
%% The new \altaffiliation can be used to indicate some secondary information
%% such as fellowships. This command produces a non-numeric footnote that is
%% set away from the numeric \affiliation footnotes.  NOTE that if an
%% \altaffiliation command is used it must come BEFORE the \affiliation call,
%% right after the \author command, in order to place the footnotes in
%% the proper location.
%%
%% Use \email to set provide email addresses. Each \email will appear on its
%% own line so you can put multiple email address in one \email call. A new
%% \correspondingauthor command is available in V6.1 to identify the
%% corresponding author of the manuscript. It is the author's responsibility
%% to make sure this name is also in the author list.
%%
%% While authors can be grouped inside the same \author and \affiliation
%% commands it is better to have a single author for each. This allows for
%% one to exploit all the new benefits and should make book-keeping easier.
%%
%% If done correctly the peer review system will be able to
%% automatically put the author and affiliation information from the manuscript
%% and save the corresponding author the trouble of entering it by hand.
\date{Performed October 7-8, 2016 and Submitted \today}


\author{Joseph Monroy}

\author{Yogesh Mehta}




%% Note that the \and command from previous versions of AASTeX is now
%% depreciated in this version as it is no longer necessary. AASTeX 
%% automatically takes care of all commas and "and"s between authors names.

%% AASTeX 6.1 has the new \collaboration and \nocollaboration commands to
%% provide the collaboration status of a group of authors. These commands 
%% can be used either before or after the list of corresponding authors. The
%% argument for \collaboration is the collaboration identifier. Authors are
%% encouraged to surround collaboration identifiers with ()s. The 
%% \nocollaboration command takes no argument and exists to indicate that
%% the nearby authors are not part of surrounding collaborations.

%% Mark off the abstract in the ``abstract'' environment. 
\begin{abstract}
Our goal for this experiment was to successfully detect a Jupiter-type exoplanet, and from this to see whether it has the known mid-transit time and transit duration. Also, from the depth of the transit we want to get the ratio between the radii of the planet and the star. In our experiment we choose to view the star HD 209458 while its planet HD 209458b, also known as Osiris, was transiting. 

\end{abstract}

\section{Introduction (JM)} \label{sec:intro}
It has only been within the past decade and a half that we have been able to detect planets around other stars. This has been a major breakthrough in all fields of astronomy because it allows us to see what kind of planets form around different types of stars. Also, it allows us to speculate more on whether we are alone in the universe or not. The most recently discovered method used to detect exoplanets is through transit photometry. This method allows us to point our telescope to a star and see if theres a change in observed flux. Depending on the duration of the observed flux from the star we can determine if there is a transiting body. 

This is the method we used in our experiment and it has proven very successful over the years for hundreds of other astronomers. Although it has already been confirmed that there is an exoplanet around HD 209458, our experiment was significant because it allowed us to see first hand how powerful the method of transit photometry is. By just knowing the dip in relative flux, we can determine the ratio between the radii of the planet and star. Then from knowing the radius of the host star, we can determine the radius of the exoplanet. This is the reason why this method is so powerful: it allows us to learn the properties of planets hundreds of light years away without even viewing or seeing the planet. 

\section{Observations (JM)}
We decided to observe HD 209458 because of the unique characteristics of the system. The star is of G type making it very similar to our own star, having a mass of 1.148 solar masses and a radius of 1.203 solar radii. The planet has a radius 1.38 times that of Jupiter, while orbiting and nearly one eighth the orbital radius of Mercury to the Sun. This makes Osiris a hot Jupiter type planet because its the size of Jupiter but extremely close to it's star. This type of system is a great candidate for exoplanet detection through transit photometry. This is because the star is of average size, while having a huge Jupiter size planet orbiting it. This makes detecting a dip in the relative flux when the planet transits a lot easier. The only negative about this system is the orbital radius of Osiris, because the closer a planet is to its star the harder it will be to detect it. Although it makes up for it by the size of Osiris and the relatively bright star of 7.65 apparent magnitute.

We were able to observe HD 209458 and its transiting planet by using a 14-inch Meade LX200-ACF telescope with a SBIG STL-1001E CCD camera attached. This yields a field of view of 26 arcminutes for the exposures. All observations were held at the Mt. Stony Brook Observatory. The weather that night was mostly clear with some thin overcast clouds in the beginning of observations. By the end of the experiment the sky was clear of clouds. The seeing was mostly good throughout the night as well with there being very little wind. We calculated the total transit duration to be 3.54 hours, with a mid-transit time of October $8^{\text{th}}$, 2016 at 02:23:23 UTC. The transit duration was calculated using the following equation:
\begin{equation}
T_{duration} = \frac{P}{\pi}\arcsin\left(\frac{\sqrt{(R_*+R_P)^{2}-(bR_*)^{2}}}{a}\right)
\end{equation}
Where $P$, $a$ and $b$ are all the period, semi-major axis and impact parameter of the planet respectively, and $R_*$ and $R_P$ are the radii of the star and planet respectively. The mid-transit time was calculated from knowing the period of the planet, and knowing a specific time when the transit began in the past. From this information we simply just have to keep adding the period time to the time of transit until we reach current day.

We first tried to find the star by going to it's right accession and declination, and made sure that the star we found was HD 209458 by looking at our finder charts. Once we confirmed the star we made sure it was centered on the cross hairs of the eyepiece. We then made sure the CCD camera was connected correctly to the telescope and the program CCDSoft, which is used for adjusting the exposure times and other settings of the CCD. We first tried to take four exposures of our star with an exposure time (exp. time) of three seconds each. This resulted in counts that were too high because the star was focused too well on the CCDSoft. High counts are bad because the pixels on the CCD can be oversaturated at around 60,000 coutnts and not yield good data. So for the next exposure we defocused the star, while increasing the exposure time to five seconds, which resulted in lower counts around 20,000. We kept this exposure time for the rest of the science exposures. By the end of the experiment we took 1,461 science exposures, all with an exposure time of five seconds. After this we started to take our calibration data, starting with the darks. In order to take the darks we put the telescope cap back on so we have complete darkness. We then took 19 dark exposures each of five seconds, then we took out flats. We took the cap off and pointed the telescope towards a equally lit part of the dome. At first we took 19 flats each of five seconds, but it turned out the counts were too high. So we lowered the exposure time to three seconds and lowered the lights in the observatory. This yielded in lower counts and better flat fields. Finally we took another set of flats with a shorter exposure time of one second. This is so later we can create a bad pixel mask so we know where the bad pixels are on the CCD. 


\section{Data Reduction}
\section{Data Analysis and Results}
\section{Discussion}
\section{Conclusion}

\end{document}